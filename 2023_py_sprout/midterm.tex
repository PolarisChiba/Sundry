\documentclass[11pt,a4paper]{article}
\usepackage{indentfirst}
\usepackage{amssymb}
\usepackage{subcaption}
\usepackage{graphicx}
\usepackage{longtable}
\usepackage{fancyhdr}
\usepackage{xeCJK}
\usepackage{amsmath}
\usepackage{amssymb}
\usepackage{ulem}
\usepackage{xcolor}
\usepackage{fancyvrb}
\usepackage{listings}
\usepackage{soul}
\usepackage{hyperref}
    \lstdefinestyle{C}{
        language=C, 
        basicstyle=\ttfamily\bfseries,
        numbers=left, 
        numbersep=5pt,
        tabsize=4,
        frame=single,
        commentstyle=\itshape\color{brown},
        keywordstyle=\bfseries\color{blue},
        deletekeywords={define},
        morekeywords={NULL,bool}
    }   
 
\setCJKmainfont{Noto Sans Mono CJK TC}
 
\voffset -20pt
\textwidth 410pt
\textheight 650pt
\oddsidemargin 20pt
\newcommand{\XOR}{\otimes}
\linespread{1.2}\selectfont
 
\pagestyle{fancy}
\lhead{2023 資訊之芽Python班階段考}
 
\begin{document}
 
\begin{center}
\section*{Problem F. 線性遞迴}
\end{center}
 
\section*{Description}
 
在此題中,我們稱呼線性遞迴是滿足以下條件的數列:

$$
F_i=\begin{cases}
1 & \text{If } 1\le i \le m \\
a_1\times F_{i-1} + a_2\times F_{i-2}+\cdots + a_m\times F_{i-m} & \text{Otherwise}
\end{cases}
$$

舉例來說,對於費氏數列來說,就是 $m=2$、$a_1 = a_2 = 1$ 的線性遞迴數列。

現在給你 $a_1\sim a_m$,還有一個正整數 $n$,請計算 $F_n$。
 
\section*{Input}
 
輸入包含兩行。

第一行會有一個數列,代表 $a_1, a_2, \cdots, a_m$。

保證此數列長度 $\le 5$、並且 $0\le a_i\le 10$。

第二行會有一個正整數,代表 $n$,保證 $1\le n\le 20$。
 
\section*{Output}
 
請輸出 $F_n$。

\section*{Sample 1}
\begin{longtable}[!h]{|p{0.5\textwidth}|p{0.5\textwidth}|}
\hline
\textbf {Input} & \textbf {Output} \\
\hline
\parbox[t]{0.5\textwidth}
{ \tt
% input
1 1\\
21\\
 
} &
\parbox[t]{0.5\textwidth}
{ \tt
%output
21\\

} \\
\hline
\end{longtable}

\section*{Sample 2}
\begin{longtable}[!h]{|p{0.5\textwidth}|p{0.5\textwidth}|}
\hline
\textbf {Input} & \textbf {Output} \\
\hline
\parbox[t]{0.5\textwidth}
{ \tt
% input
1 2 3\\
10\\

} &
\parbox[t]{0.5\textwidth}
{ \tt
%output
861\\

} \\
\hline
\end{longtable}
 
\end{document}