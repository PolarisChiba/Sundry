%架構: 其他檔案放入documentclass,並input本template。請參考Museum/Adv-Example.tex
%引入較多package,功能全面(但依賴的東西也比較多)
%Template by Saintan

%%%引入Package%%%
%欲查詢各package詳細使用方法,參見https://ctan.org/pkg/<package名稱>
%如https://ctan.org/pkg/ulem,並閱讀Package documentation
    %% font & format %%
\usepackage[margin=2cm]{geometry} % 上下左右距離邊緣2cm
\usepackage{fontspec}   % 加這個就可以設定字體(我不會用)
\usepackage{type1cm}	% 設定fontsize用(我不會用)
\usepackage{titlesec}   % 設定section等的字體(我不會用)
\usepackage{fancyhdr}   % 頁首頁尾
\usepackage{multicol}   % 多欄 \multicols
\usepackage{ulem}       % 字加裝飾 (underline+emphasis)(我不會用)
    %% Math & graphics %%
\usepackage{amsmath,amsthm,amssymb} % 引入 AMS 數學環境
\usepackage{relsize}
\usepackage{dsfont}
\usepackage{mathtools}  % 新一代數學工具,修正一點amsmath的bug
\usepackage{graphicx}   % 圖形插入用,\rotatebox等。
\usepackage{yhmath}     % BIG math symbol,如matrix用的超大圓括弧、角括弧
\usepackage{tikz-cd}    % 含tikz,可以多畫diagram類的東西
    %% Enhancement %%
\usepackage{titling}    % 加強 title 功能(我不會用)
\usepackage{tabularx}   % 加強版 table(我不會用)
\usepackage{enumerate}   % 加強版enumerate, itemize, list
\usepackage[showzone=true, showseconds=true]{datetime2} %加強版時間格式
\usepackage[square, comma, numbers, super, sort&compress]{natbib}  % cite加強版(我不會用)
    %% Logos & symbols %%
%\usepackage{textcomp}  % 可以打出許多好用的符號,Overleaf已預載
\usepackage{academicons}% 可以打出arXiv, Academia, Overleaf等Logo 
\usepackage{listings}

\usepackage{wasysym}    % 可以打出各種怪符號如 \sun, \smiley, \eighthnote
\usepackage{dsfont}     % 另一種雙標線字體 \mathds,多了\mathds{1 h k}
\usepackage[scr]{rsfso} % 草寫字體 \mathscr,請只使用其中一個
%\usepackage{BOONDOX-uprscr} % 另一個草寫字體 \mathscr,請只使用其中一個
    %% Chinese/Japanese/Korean%%%
\usepackage[CJKmath=true,AutoFakeBold=3,AutoFakeSlant=.2]{xeCJK}% XeLaTeX 中日韓
\usepackage{CJKulem} %中日韓裝飾
% \usepackage{setspace}
\usepackage{minted}

\setCJKmainfont{AR PL KaitiM Big5} %一般字體(\textrm)
\setCJKsansfont{AR PL KaitiM Big5} %無襯線字體(\textrf)
\setCJKmonofont{AR PL KaitiM Big5} %打字機字體(\texttt)
\newCJKfontfamily\Kai{標楷體} 
\newCJKfontfamily\Hei{微軟正黑體}
\newCJKfontfamily\NewMing{新細明體}
\XeTeXlinebreaklocale "zh"
    %% Other package %%
\usepackage{pgf}    %For random
\usepackage[unicode=true, pdfborder={0 0 0}, bookmarksdepth=-1]{hyperref} % ref加強版,請儘量把hyperref放在最後一個引入的package

\usepackage[autostyle]{csquotes}  

%%%頁面設定%%%
\setlength{\headheight}{15pt}  %with titling
\setlength{\droptitle}{-1.5cm} %title 與上緣的間距
\parindent=24pt %設定縮排的距離

%%%證明、結論、定義等等的環境,derived from CKMSG%%%
\newtheoremstyle{mystyle}% 自定義Style
  {6pt}{15pt}%       上下間距
  {}%               內文字體
  {}%               縮排
  {\bf}%            標頭字體
  {.}%              標頭後標點
  {1em}%            內文與標頭距離
  {}%               Theorem head spec (can be left empty, meaning 'normal')

% 改用粗體,預設 remark style 是斜體
\theoremstyle{mystyle}	% 定理環境Style
\newtheorem{thm}{Theorem}[section]
\newtheorem{df}[thm]{Definition}
\newtheorem{ex}[thm]{Example}
\newtheorem{exs}[thm]{Exercise}
\newtheorem{cl}[thm]{Corollary}
\newtheorem{pp}[thm]{Property}
\newtheorem{prop}[thm]{Proposition}
\newtheorem{lm}[thm]{Lemma}
\newtheorem{pr}{Problem}
\renewcommand{\proofname}{\bf Proof:\\} %修改Proof 標頭,Proof有內建
\newtheorem*{rmk}{Remark}
\newtheorem*{clm}{Claim}

\newcommand{\resetcounters}
    {
    \setcounter{section}{0}
    \setcounter{thm}{0}
    }

%%%重定義一些command%%%
\newcommand{\st}{\mbox{ such that }}
\newcommand{\N}{\mathbb{N}}
\newcommand{\Z}{\mathbb{Z}}
\newcommand{\Q}{\mathbb{Q}}
\newcommand{\R}{\mathbb{R}}
\renewcommand{\C}{\mathbb{C}}
\newcommand{\F}{\mathbb{F}}
\DeclareMathOperator{\im}{Im}
\DeclareMathOperator{\trace}{tr}
\DeclareMathOperator{\vsspan}{span}
\DeclareMathOperator{\image}{Im}
\DeclareMathOperator{\rank}{rank}
\DeclareMathOperator{\adj}{adj}
\DeclareMathOperator{\ch}{ch}

\pagestyle{fancy}  % fancy: fancyhdr

%use with fancyhdr
\lhead{\leftmark}
\chead{}
\rhead{第四屆 全國高中數理科學競賽培訓營資訊組}
\cfoot{}
\rfoot{\thepage}
\renewcommand{\headrulewidth}{0.4pt}
\renewcommand{\footrulewidth}{0.4pt}


%%%\doublespacing